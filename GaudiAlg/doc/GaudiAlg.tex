\documentclass{lhcbnote}

\usepackage{times}
\usepackage{listings}


\newcommand{\bfsc}         {\scshape\bfseries}
\newcommand{\bftt}         {\ttfamily\bfseries}
\newcommand{\bfit}         {\itshape\bfseries}
\newcommand{\scbf}         {\scshape\bfseries}
\newcommand{\slbf}         {\slshape\bfseries}
\newcommand{\itbf}         {\itshape\bfseries}
\newcommand{\ttbf}         {\ttfamily\bfseries}

\renewcommand{\sc  }       {\scshape}
\renewcommand{\sl}         {\slshape}
\renewcommand{\it}         {\itshape}
\renewcommand{\tt}         {\ttfamily}


\begin{document}

\lstset{language=[ANSI]C++}
%\lstset{language=[Visual]C++}
\lstset{indent=15mm}
\lstset{labelstep=1}
\lstset{labelstyle=\tt\tiny}


\doctyp{Internal Note}
\dociss{1}
\docrev{3}
%\docref{}
\doccre{January 15, 2004}
\docmod{\today}

\title{Proposal for extension of {\bftt{GaudiAlgorithm}} and {\bftt{GaudiTool}} classes}
\author{Vanya Belyaev\footnote{\bftt{E-mail: Ivan.Belyaev\@itep.ru}}}
\maketitle

\chapter{Introduction}

The existing classes {\bftt{GaudiAlgorithm}} from the  
package {\scbf{GaudiAlg}} and 
{\bftt{GaudiTool}} from the package 
{\scbf{GaudiTools}} provide some extension of 
the functionality of standard classes {\bftt{Algorithm}}
and {\bftt{AlgTool}} from {\scbf{GaudiKernel}} package. 

A large number of extension of 
class {\bftt{Algorithm}} are widely used in 
LHCb, e.g. {\bftt{CaloAlgorithm}} class from 
{\scbf{Calo/CaloKernel}} package, {\bftt{LoKi::Algo}} 
class from {\scbf{Tools/LoKi}}, {\bftt{TrAlgorithm}} 
class from {\scbf{Tr/TrKernel}}, 
{\bftt{RichAlgBase}} class from {\scbf{Rich/RichUtils}},
{\bftt{ITCheckAlgorithm}} from 
{\scbf{IT/ITCheckers}}, 
{\bftt{OTMonitorAlgorithm}} from {\scbf{OT/OTAlgorithms}}.
Also large number of extension for {\bftt{AlgTool}} class 
are used in LHCb software: 
{\bftt{CaloTool}} from {\scbf{Calo/CaloKernel}},
{\bftt{GiGaTool}} from {\scbf{Sin/GiGa}}, 
{\bftt{RichToolBase}} from 
{\scbf{Rich/RichUtils}}. An activity of trigger 
group in creation of {\scbf{Gaudies}} package and 
{\bftt{TrgAlgorithm}} base class is also need to be mentioned here. 

Existing {\bftt{GaudiAlgorithm}} and 
{\bftt{GaudiTools}} classes were designed 
just to enrich the base classes 
{\bftt{Algorithm}} and {\bftt{AlgTool}} 
with some additional and more user-friendly 
functionality namely
\begin{itemize}
 \item error and warning printout with error count 
   and full statistics of errors and warnings  
 \item easy retrieve of objects from Gaudi Transient Store 
 \item easy register of objects into Gaudi Transient Store 
 \item easy and safe location of services and tool 
       with user-friendly semantics 
 \item easy access to ``standard'' services 
\end{itemize} 

In this document the existing functionality is reviewed 
with possible extensions to 
algorithms, specialized for dealing with histograms 
{\bftt{GaudiHistoAlg}} and N-tuples 
{\bftt{GaudiTupleAlg}}. The first extension class 
could act as a good base class for monitoring algorithms, 
and the second extension class is a good base class 
e.g. for analysis algorithms.   

\chapter{The extended functionality of {\bftt{GaudiAlgorithm}} } 

\section{Error handling}

{\bftt{GaudiAlgorithm}} class is equipped with 
few methods for easy printout of errors and warnings:

\begin{scriptsize}
 \begin{lstlisting}{}

  StatusCode 
  Error     
  ( const std::string& msg , 
    const StatusCode   st  = StatusCode::FAILURE ,
    const size_t       mx  = 10                  ) const ;
  
  StatusCode 
  Warning   
  ( const std::string& msg , 
    const StatusCode   st  = StatusCode::FAILURE ,
    const size_t       mx  = 10                  ) const ;

 \end{lstlisting}
\end{scriptsize}

The error/warning messages are printed and counted.
The overall statistics of errors and warnings are 
printed at the end of the job. The last argument of 
these commands defines the maximal number of identical 
printouts.

The usage of these methods is straightforward: 

\begin{scriptsize}
 \begin{lstlisting}{}

    StatusCode MyAlg::execute() 
    {
      ....
      if ( 0 >  a ) { return Error("value of 'a' is negative!") ; }
      if ( 0 == a ) { return Warning("value of 'a' is zero!" )  ; }
      if ( sc.isFailure() ) 
          { return Error("something wrong here",sc) ; }
      ....
    };

 \end{lstlisting}
\end{scriptsize}

The result table, printed at the end of the job looks like 

\begin{scriptsize}
  \begin{lstlisting}{}

    MyAlgName SUCCESS Exceptions/Errors/Warnings statistics:  0/2/1
    MyAlgName SUCCESS #ERRORS    = 118 Message='value of 'a' is negative!'
    MyAlgName SUCCESS #ERRORS    = 400 Message='something wrong here'      
    MyAlgName SUCCESS #WARNINGS  = 327 Message='value of 'a' is sero!'    

  \end{lstlisting}
\end{scriptsize}

Here the first line contains an inforomation about number of different types 
of exceptions, errors and warning, and each subsequent line corresponds to 
certain exception/error/warning and provides the information about error 
message and occurance number.


The second argument allows to assign easily the 
meaningful values for return error codes, e.g.

\begin{scriptsize}
 \begin{lstlisting}{}

    StatusCode MyAlg::execute() 
    {
      ....
      if( 0 >  a ) 
        { return Error ( " 'a' is negative!", StatusCode( 300 ) ) ; }
      if( 0 == a ) 
        { return Error ( " 'a' is zero!"    , StatusCode( 301 ) ) ; }
      StatusCode sc = someMethod() 
      if( sc.isFailure() ) 
        { return Error ( " The error from 'someMethod()'" , sc ) ; }
      ....
    };

 \end{lstlisting}
\end{scriptsize}

The third argument allows to limit the number of 
error printouts:

\begin{scriptsize}
 \begin{lstlisting}{}

    StatusCode MyAlg::execute() 
    {
      for( int i = 0 ; i < 100000 ; ++i ) 
      {
      if( 0 == i%2  ) 
        { Warning ( " the first  warning " , StatusCode( 300 ) , 20 ) ; }
      if( 1 == i%2  ) 
        { Warning ( " the second warning " , StatusCode( 301 ) , 30 ) ; }
      ....
    };

 \end{lstlisting}
\end{scriptsize}

For this example the total number of printouts from the in log-file
will be only 20 for the first warning message and only 30 for the second 
warning message. Wihle at the end of the job, the overall 
number of warnings of the first type and second type will 
be evaluated correctly as 
$20\times50000\times{\mathcal{N}}_{\mathrm{ev}}$ and 
$30\times50000\times{\mathcal{N}}_{\mathrm{ev}}$ 
correspondingly and printed as a table at the end of the job.


Few additional methods are closely related to 
{\bftt{GaudiAlgorithm::Error}} and 
{\bftt{GaudiAlgorithm::Warnings}} methods:

\begin{scriptsize}
 \begin{lstlisting}{}

  StatusCode Print     
  ( const std::string& msg , 
    const StatusCode   st  = StatusCode::SUCCESS ,
    const MSG::Level   lev = MSG::INFO           ) const ;
  
  inline 
  StatusCode
  Assert 
  ( const bool         ok                            , 
    const std::string& message = ""                  , 
    const StatusCode   sc      = StatusCode::FAILURE ) const;
  
  inline 
  StatusCode 
  Assert 
  ( const bool         ok                            , 
    const char*        message                       ,
    const StatusCode   sc      = StatusCode::FAILURE ) const;
  
  StatusCode 
  Exception 
  ( const std::string    & msg                        ,  
    const GaudiException & exc                        , 
    const MSG::Level       lvl = MSG::FATAL           ,
    const StatusCode       sc  = StatusCode::FAILURE ) const ;
  
  StatusCode 
  Exception 
  ( const std::string    & msg                        ,  
    const std::exception & exc                        , 
    const MSG::Level       lvl = MSG::FATAL           ,
    const StatusCode       sc  = StatusCode::FAILURE ) const ;
  
  StatusCode 
  Exception 
  ( const std::string& msg = "no message"        ,  
    const MSG::Level   lvl = MSG::FATAL          ,
    const StatusCode   sc  = StatusCode::FAILURE ) const ;

 \end{lstlisting}
\end{scriptsize}

The exceptions, thrown by these methods are also counted 
and printed at the end of the job. The error message is 
printed BEFORE exception is thrown. The usage of these 
methods is straightforward: 

\begin{scriptsize}
 \begin{lstlisting}{}

    StatusCode MyAlg::execute() 
    {
      // check the assertion 
      Assert( a > 0 , " value of 'a' is not positive!" );
      ....
      try { ... } 
      catch ( const GaudiException& exc ) 
      { // catch the expected exception
	Exception( " Exception is catched and re-thrown" , exc ) ;
      }
      // thrown exception (return is 'fictive' here)
      if( a < 0 ) { return Exception(" 'a' is negative !" );}
      ...
    };

 \end{lstlisting}
\end{scriptsize}

An easy macro {\bftt{ALG\_ERROR}} is defined 
to supply the error message with file name and line number 
(such trick often is very useful to locate the exact place where 
the problem occurs)

\begin{scriptsize}
 \begin{lstlisting}{}

    StatusCode MyAlg::execute() 
    {
      ...
      StatusCode sc = StatusCode::FAILURE ;
      if( a < 0 ) { return ALG_ERROR(" 'a' is negative !" , sc ) ; }
    };

 \end{lstlisting}
\end{scriptsize}

 Here the error message will be extended with information
about the source file name and line number.
The result table, printed at the end of events looks like 

\begin{scriptsize}
  \begin{lstlisting}{}

    MyAlgName SUCCESS Exceptions/Errors/Warnings statistics:  0/1/0
    MyAlgName SUCCESS #ERRORS = 118 Message='value of 'a' is negative! \ 
                                   [ in file ../src/MyAlg.cpp line #115 ]'
  \end{lstlisting}
\end{scriptsize}


\section{Predefined message streams}
Markus Frank\footnote{\bftt{E-mail: Markus.Frank@cern.ch}} has kindly proposed to have few predefined message 
streams with fixed verbosity levels to be always available:

\begin{scriptsize}
  \begin{lstlisting}{}

    MsgStream& always  () const ;
    MsgStream& fatal   () const ;
    MsgStream& err     () const ;
    MsgStream& warn    () const ;
    MsgStream& info    () const ;
    MsgStream& debug   () const ;
    MsgStream& verbose () const ;

  \end{lstlisting}
\end{scriptsize}

The usage of these predefined streams 
allows to minimize the visible difference between 
standard {\scbf{STL}} output streams 
{\bftt{std::cout}}, {\bftt{std::cerr}},
{\bftt{std::clog}}, and e.g. {\bftt{GaudiAlgorithm::err()}}

\begin{scriptsize}
 \begin{lstlisting}{}

    StatusCode MyAlg::execute() 
    {
      ...
      if( 0 >  a ) { err()     << " 'a' is negative !" << endmsg    ; }
      if( 0 == a ) { warn()    << " 'a' is zero !"     << endmsg    ; }
      // to be compared with 
      if( 0 >  a ) { std::cerr << " 'a' is negative !" << std::endl ; }
      if( 0 == a ) { std::cout << " 'a' is zero !"     << std::endl ; }

    };

 \end{lstlisting}
\end{scriptsize}

The certain care need to be taken since this approach have 
few fragile features, but in general its simplicity and 
transparency makes it to be very attractive. 

\section{Control over the debug actions}
Additional features are proposed to import from 
{\bftt{RichAlgBase}} and {\bftt{RichToolBase}} classes
({\scbf{RichUtils}} package): 

\begin{scriptsize}
 \begin{lstlisting}{}

  /// Test the printout level
  inline bool        msgLevel ( MSG::Level level ) const ;

  /// Return message level setting
  inline MSG::Level  msgLevel ()                  const ;

 \end{lstlisting}
\end{scriptsize}

Such methods allows to have more control 
over the printout levels and to write more efficient debug printout, e.g.

\begin{scriptsize}
 \begin{lstlisting}{}

    StatusCode MyAlg::execute() 
    {
      ...
      if( msgLevel( MSG::DEBUG ) ) 
      {
	// evaluate debug information,
	// construct debug message and print it 
      } 

    };

 \end{lstlisting}
\end{scriptsize}


\section{Data handling}

To simplify the data retrieval from Gaudi Transient Store 
{\bftt{GaudiAlgorithm}} provides 2 methods to retrieve 
data and on method to register the data in Gaudi Stores :

\begin{scriptsize}
 \begin{lstlisting}{}

  template<class TYPE>
  TYPE*   get    ( IDataProviderSvc*  svc        , 
                   const std::string& location   ) const ;

  template<class TYPE>
  TYPE*   get    ( const std::string& location   ) const ;

  template<class TYPE>
  TYPE*   getDet ( const std::string& location   ) const ;

  StatusCode put ( DataObject*        object  , 
                   const std::string& address ) const ;
  
 \end{lstlisting}
\end{scriptsize}

The second method is just a shortcut for the first 
method for retrieval data from Gaudi Event Data Store, 
which is probably is the most frequently use-case.
The third method is the shortcut to retvieve objects form 
Gaudi Detector Data Store.  
If data are not available, or corrupted (e.g. have different type), 
an exception is thrown. {\bf{The methods guarantees to 
    return the valid data}}. One can register the data 
only in the Gaudi Event Transient store.

The usage is illustrated by following listing:

\begin{scriptsize}
 \begin{lstlisting}{}

    StatusCode MyAlg::execute() 
    {
      ...
      MCParticles* p = get<MCParticles>( eventSvc() , MCParticleLocation::Default ) ; 
      MCVertices*  v = get<MCVertices>( MCVertexLocation::Default );
      const IDetectorElement* det = getDet<IDetectorElement>( "/dd/Structure/LHCb");
      Particles* ps = new Particles() ;
      put( Particles , "Phys/MyParticles");
    };

 \end{lstlisting}
\end{scriptsize}


\section{Handling of tools and services}

{\bftt{GaudiAlgorithm}} class is equipped with few method for 
easy location of services and tools

\begin{scriptsize}
 \begin{lstlisting}{}

  template<class TOOL>
  TOOL*    tool ( const std::string& type           , 
                  const std::string& name           , 
                  const IInterface*  parent  = 0    , 
                  bool               create  = true ) const ;

  template<class TOOL>
  TOOL*    tool ( const std::string& type          , 
                  const IInterface*  parent = 0    , 
                  bool               create = true ) const ;

  template<class SERVICE>
  SERVICE* svc  ( const std::string& name           , 
                  const bool         create = false ) const ;

 \end{lstlisting}
\end{scriptsize}

{\bf{Services and tools, located with these methods should not 
be manually released}}. They will be released automatically 
at algorithm finalization. The methods thrown exceptions if
service or tool can not be retrieved, and guarantee to
return a valid pointer to service/tool. 
The semantics is very similar to semantics of 
{\bftt{GaudiAlgorithm::get}} method:

\begin{scriptsize}
 \begin{lstlisting}{}

    StatusCode MyAlg::initialize() 
    {
      ...
      IMyTool* m1 =  tool<IMyTool>( "MyToolType" ,  "MyToolName" , this ) ;
      IMyTool* m2 =  tool<IMyTool>( "MyToolType/MyToolName"      , this ) ;
      IMyTool* m3 =  tool<IMyTool>( "MyToolType"                 , this ) ;
      IMySvc*  s1 =  svc<IMySvc>  ( "MyService"                  , true ) ;
    };

 \end{lstlisting}
\end{scriptsize}

Two methods for forced manual release of tools/services, 
acquired with {\bftt{GaudiAlgorithm::tool}} and 
{\bftt{GaudiAlgorithm::svc}} methods, are provided just to 
take into account the case when components are only needed 
for a short period, such as during initialisation:

\begin{scriptsize}
  \begin{lstlisting}{}
    
    StatusCode release ( const IAlgTool* tool ) const ;
    StatusCode release ( const IService* svc  ) const ;
    
  \end{lstlisting}
\end{scriptsize}

In this case the tool/service will be immediately released and 
one can avoid the hanging of tool/service till the end of 
the job.

\begin{scriptsize}
  \begin{lstlisting}{}

    StatusCode MyAlg::initialize() 
    {
      const IMyTool* m3 = tool<IMyTool>( "MyToolType/MyToolName" , this ) ;
      m3 -> doSomeJobWithTheGreatSuccess();
      release ( m3 ) ;
    };

  \end{lstlisting}
\end{scriptsize}

If tool or service was not previously aquired by 
{\bftt{GaudiAlgorthm::tool}} or {\bftt{GaudiAlgorthm::svc}}
methods, the corresponding invokation of 
{\bftt{GaudiAlgorithm::release}} has no effect.



\section{Useful and convenient  shortcuts for some standard methods}

For convenience and coherency it is proposed 
to introduce few shortcuts for standard methods, provided by 
base class {\bftt{Algorithm}}

\begin{scriptsize}
 \begin{lstlisting}{}

   // synonym for Algorithm::eventSvc()   
   IDataProviderSvc*  evtSvc() const ; 
   // synonym for Algorithm::serviceLocator()
   ISvcLocator*       svcLoc() const ; 
   IIncidentSvc*      incSvc() const ;

 \end{lstlisting}
\end{scriptsize}

Together with standard methods {\bftt{Algorithm::detSvc}} and 
{\bftt{Algorithm::msgSvc}} they provide nice 6-symbol acronyms.


\section{Other useful features}
For the ouput level {\bftt{MSG::DEBUG}}
{\bftt{GaudiAlgorithm::initialize}} method performs the 
printout of all properties of algorithm:

\begin{scriptsize}
  \begin{lstlisting}{}

    MyAlgName DEBUG  List of ALL properties of MyAlg/MyAlgName   #properties = 38
    MyAlgName DEBUG Property ['Name': Value] = "HistoOffSet": 0
    MyAlgName DEBUG Property ['Name': Value] = "CheckForNaN": 1
    MyAlgName DEBUG Property ['Name': Value] = "ProduceHistos": 1
    MyAlgName DEBUG Property ['Name': Value] = "Output": ""
    MyAlgName DEBUG Property ['Name': Value] = "Input": ""
    MyAlgName DEBUG Property ['Name': Value] = "AuditFinalize": 0
    MyAlgName DEBUG Property ['Name': Value] = "AuditExecute": 1
    MyAlgName DEBUG Property ['Name': Value] = "AuditInitialize": 0
    MyAlgName DEBUG Property ['Name': Value] = "ErrorCount": 0

  \end{lstlisting}
\end{scriptsize}


\section{Implementation policy for derived classes}
If concrete algorithm requires the implementation of 
the specific {\bftt{initialize}} or 
{\bftt{finalize}} methods they {\bftt{must}} be implemented 
in a following way:

\begin{scriptsize}
 \begin{lstlisting}{}

   class MyAlg: public GaudiAlgorithm { ... } ;

   // standard constructor 
   MyAlg::MyAlg( const std::string& name , 
                 ISvcLocator*       svc  ) 
   : GaudiAlgorithm( name , svc ) 
   , ....
    
   // initialization
   StatusCode MyAlg::initialize() 
   {
     // initialize the base class 
     StatusCode sc = GaudiAlgorithm::initialize() ;
     if( sc.isFailure() ) 
     { return Error("Base class is not initialized",sc);}
     // put specific code here 
     ...
     return StatusCode::SUCCESS ;
   };
   
   // initialization
   StatusCode MyAlg::finalize() 
   {
     // put specific code here 
     ...
     return GaudiAlgorithm::finalize();
   };

 \end{lstlisting}
\end{scriptsize}




It is assumed that {\bftt{GaudiTool}} class is equipped with 
all these methods (error handling, data handling, 
tools/services handling and shortcuts).

\chapter{The functionality, proposed for {\bftt{GaudiHistoAlg}}}
The class {\bftt{GaudiHistoAlg}} inherits all features 
from {\bftt{GaudiAlgorithm}} base class. In addition it gets
the methods for easy handling of histograms:

\begin{scriptsize}
 \begin{lstlisting}{}
 
  AIDA::IHistogram1D*  plot 
  ( const double        value        ,
    const std::string&  title        ,
    const double        low          ,
    const double        high         ,
    const unsigned long bins   = 100 ,
    const double        weight = 1.0 ) const ;

  AIDA::IHistogram1D*  plot      
  ( const double        value        ,
    const HistoID&      ID           , 
    const std::string&  title        ,
    const double        low          ,
    const double        high         ,
    const unsigned long bins   = 100 ,
    const double        weight = 1.0 ) const ;

 \end{lstlisting}
\end{scriptsize}

{\bf{For usage of these method one do not need to book the histogram.}} 
Booking is performed on demand. The unique integer histogram ID 
for the first method is generated automatically. 
For the the second method (which is more ``HBOOK''-oriented) 
the assignment of integer ID is under the user control:

\begin{scriptsize}
 \begin{lstlisting}{}
   
   StatusCode MyAlg::execute() 
   {
     double pt = ... ;
     plot( pt/GeV           , "Transverse momentum in GeV/c" , 0  , 10 ) ;
     plot( sin(pt)/GeV , 15 , "Sine of PT"                   , -1 ,  1 ) ;
   };
   
 \end{lstlisting}
\end{scriptsize}

As it was emphasized histogram will be booked automatically on-demand.
But in addition one can use {\bftt{GaudiHistoAlg::book}} method explicitely to book 
the histograms:

\begin{scriptsize}
 \begin{lstlisting}{}

    StatusCode MyAlg::initialize() 
    {
      book ( "Transverse momentum in GeV/c" , 0 , 10 );
      AIDA::IHistogram* h15 = 
      book ( 15 , "Sine of PT" , -1 , 1 ) ;
    };
    
 \end{lstlisting}
\end{scriptsize}

Both {\bftt{GaudiHistoAlg::book}}
and {\bftt{GaudiHistoAlg::plot}} methods return 
the pointer to  {\bftt{ AIDA::IHistogram1D}} 
class which can be used for explicit manipulations 
with histograms.

In addition to simple methods of putting {\bftt{double}}
value to the histogram, few sophisticated methods for 
efficient filling of the  histograms with container-organized data are
provided:

\begin{scriptsize}
  \begin{lstlisting}{}
    
    template <class OBJECT,class FUNCTION>
    IHistogram1D*  plot 
    ( FUNCTION            func         ,
      OBJECT              first        , 
      OBJECT              last         , 
      const std::string&  title        ,
      const double        low          ,
      const double        high         ,
      const unsigned long bins   = 100 ) const ;

    template <class OBJECT,class FUNCTION>
    IHistogram1D*  plot 
    ( FUNCTION            func         ,
      OBJECT              first        , 
      OBJECT              last         , 
      const HistoID&      ID           ,
      const std::string&  title        ,
      const double        low          ,
      const double        high         ,
      const unsigned long bins  = 100  ) const 

    template <class OBJECT,class FUNCTION,class WEIGHT>
    IHistogram1D*  plot 
    ( FUNCTION            func         ,
      OBJECT              first        , 
      OBJECT              last         , 
      const std::string&  title        ,
      const double        low          ,
      const double        high         ,
      const unsigned long bins         , 
      WEIGHT              weight       ) const ;

    template <class OBJECT,class FUNCTION,class WEIGHT>
    IHistogram1D*  plot 
    ( FUNCTION            func         ,
      OBJECT              first        , 
      OBJECT              last         , 
      const HistoID&      ID           ,
      const std::string&  title        ,
      const double        low          ,
      const double        high         ,
      const unsigned long bins         , 
      WEIGHT              weight       ) const   

 \end{lstlisting}
\end{scriptsize}

Again , each method has 2 variants: with and without explicit 
assignment of integer identifier.
The usage of these method is illustrated by following listing:

\begin{scriptsize}
 \begin{lstlisting}{}

   StatusCode MyAlg::execute()
   {
     typedef std::vector<double> Container ; 
     Container data = ... ;
     plot( sin , data.begin() , data.end() , 
           "Plot 1: sin(data) ", -1 , 1 ); 
     plot( sin , data.begin() , data.end() , 
           "Plot 2: sin(data) with exp-weight" , -1 , 1 , 100 , exp ); 
   }; 

 \end{lstlisting}
\end{scriptsize}

Here the type of {\bftt{OBJECT}}, {\bftt{FUNCTION}} and 
{\bftt{WEIGHT}} can be arbitrary, there are only 2 limitations:
\begin{itemize}
\item {\bftt{OBJECT}} is an iterator, which supports the forward iterator protocol
({\bftt{++first}} and {\bftt{*first}}).
\item {\bftt{func(*first)}} and {\bftt{weight(*first)}} are semantically  
  valid expressions, which can be evaluated to the type, implicitely 
  convertible to {\bftt{double}}
\end{itemize}  
As an example one can consider the usage of 
function {\bftt{PT}} from {\scbf{LoKi}} package:

\begin{scriptsize}
 \begin{lstlisting}{}

   StatusCode MyAlg::execute()
   {
     Particles* p = get<Particles>( ParticleLocation::Default );
     plot( PT/GeV , p->begin() , p->end() , "Transverse momentum " , 0  , 10 ); 
   }; 

 \end{lstlisting}
\end{scriptsize}

In this example, in 2 lines one performs the location of 
container of particles
in Gaudi Event Transient Store, booking and filling the histogram with 
distribution of transverse momentum  for all particles.


The histograms go into a subdirectory in Gaudi Histogram Transient 
Store which name is configurable through properties of the algorithm, 
and the default name corresponds to the name of algorithm instance. 
The nesessary split of long names for HBOOK persistency is 
optional and performed with the control over algorithm properties. 

The method {\bftt{fullDetail}} is especially 
convinient for monitoring algorithms and is imported from 
{\bftt{ITCheckAlgorithm}} from {\scbf{ITCheckers}} package:

\begin{scriptsize}
 \begin{lstlisting}{}
   
   bool fullDetail() const ;

 \end{lstlisting}
\end{scriptsize}

The full list of algorithm properties, 
introduced for the class {\bftt{GaudiHistoAlg}}
is presented in Table~\ref{Table:AlgHistoProperties}.

\begin{table}[hbt]
  \caption[Standard properties of class {\bftt{GaudiHistoAlg}}]
          {The standard properties of class {\bftt{GaudiHistoAlg}} 
and their default values.}
\bigskip
\label{Table:AlgHistoProperties}
\begin{tabular*}{\linewidth}{@{\hspace{10mm}}l@{\extracolsep{\fill}}l@{\hspace{10mm}}}
\hline 
\hline 
Property Name & Default Value  \\
\hline
{\bftt{"ProduceHistos"}}            &  {\bftt{true}}             \\ 
{\bftt{"HistoOffSet"}}              &  {\bftt{0}}                \\
{\bftt{"SplitHistoDir"}}            &  {\bftt{true}}             \\
{\bftt{"HistoTopDir"}}              &  {\bftt{""}}               \\
{\bftt{"HistoDir"}}                 &  Algorithm instance name \\
{\bftt{"CheckForNaN"}}              &  {\bftt{true}}             \\
\hline
{\bftt{"FullDetail"}}               &  {\bftt{false}}            \\ 
\hline 
\end{tabular*} 
\end{table}

The producton of histograms can be switched on/off using the  
property {\bftt{"ProduceHistos"}}.
 For switched off production of histograms 
all {\bftt{GaudiHistoAlg::plot}} and  {\bftt{GaudiHistoAlg::book}} \ methods have no effect.
An algorithm-specific 
control is performed using the property {\bftt{"FullDetail"}}.


\section{Implementation policy for derived classes}
If concrete algorithm requires the implementation of 
the specific {\bftt{initialize}} or 
{\bftt{finalize}} methods they {\bftt{must}} be implemented 
in a following way:

\begin{scriptsize}
 \begin{lstlisting}{}

   class MyAlg: public GaudiHistoAlg { ... } ;

   // standard constructor 
   MyAlg::MyAlg( const std::string& name , 
                 ISvcLocator*       svc  ) 
   : GaudiHistoAlg( name , svc ) 
   , ....

   // standard initialzation
   StatusCode MyAlg::initialize() 
   {
     // initialize the base class 
     StatusCode sc = GaudiHistoAlg::initialize() ;
     if( sc.isFailure() ) 
     { return Error("Base class is not initialized",sc);}
     // put specific code here 
     ...
     return StatusCode::SUCCESS ;
   };
   
   // standard finalization
   StatusCode MyAlg::finalize() 
   {
     // put specific code here 
     ...
     return GaudiHistoAlg::finalize();
   };

 \end{lstlisting}
\end{scriptsize}


\chapter{The functionality, proposed for {\bftt{GaudiTupleAlg}}}
It is proposed that  class {\bftt{GaudiTupleAlg}} inherits 
all functionality of {\bftt{GaudiHistoAlg}} and gets functionality 
for easy dealing with N-tuples, 
imported from {\scbf{LoKi}} package. {\bf{One do not need to book tuple and its columns. 
Tuple and columns are booked on-demand.}}. 


For simple columns the proposed functionality is 
illustrated by following listing

\begin{scriptsize}
  \begin{lstlisting}{}

    StatusCode MyAlg::execute()
    {
      Tuple tuple = ntuple( "Unique N-tuple title-string identifier " );
      tuple->column( "Mass" , mass / GeV  );
      tuple->column( "PT"   , pt / GeV );
      tuple->column( "chi2" , chi2 );
      tuple->write() ;
    }; 

  \end{lstlisting}
\end{scriptsize}
Few columns can be filled simultaneously\footnote{For this case 
all of them need to be of type {\bftt{double}}}

\begin{scriptsize}
  \begin{lstlisting}{}

    StatusCode MyAlg::execute()
    {
      Tuple tuple = ntuple( "Unique N-tuple title-string identifier " );
      tuple->fill( "Mass,PT,chi2" , mass / GeV  , PT/GeV, chi2 );
      tuple->write() ;
    }; 

  \end{lstlisting}
\end{scriptsize}
Few methods allow to deal with N-tuples:

\begin{scriptsize}
  \begin{lstlisting}{}

    Tuple ntuple( const std::string& title                       , 
                  const CLID&        type = CLID_ColumnWiseTuple ) const ;

    Tuple ntuple( const HbookID&     ID                          , 
                  const std::string& title                       , 
                  const CLID&        type = CLID_ColumnWiseTuple ) const ;

  \end{lstlisting}
\end{scriptsize}

The basic type of column, which appears in PAW or ROOT 
are {\bftt{float}} and {\bftt{long}}. Few shortcuts are exist 
to put into N-tuple complex objects, like {\bftt{HepPoint3D}}, 
{\bftt{HepLorenzVector}}, etc.

The filling of array-like columns is also simple.

The detailed description of usage of proposed N-tuple facility see
{\scbf{LoKi}} User Guide, available as 
{\bftt{LoKi.ps}} in 
{\bftt{/afs/cern.ch/user/i/ibelyaev/doc/GaudiDoc}} directory\footnote{
Alternatively one can use files {\bftt{LoKi.pdf}},{\bftt{LoKi\_2on1.ps}}
or {\bftt{LoKi\_2on1.pfd}} files in the same directory}

The list of available tuple fill methods  
is presented in following listing\footnote{For 
row-wise N-tuples only {\bftt{column}} and 
{\bftt{fill}} methods have effect}

\begin{scriptsize}
  \begin{lstlisting}{}

    // primitive columns 
    StatusCode column ( const std::string&       name   , 
                        const long               var    ) ;
    StatusCode column ( const std::string&       name   , 
                        const bool               var    ) ;
    StatusCode column ( const std::string&       name   , 
                        const double             var    ) ;
    StatusCode column ( const std::string&       name   , 
                        const long               var    , 
                        const long               minv   , 
                        const long               maxv   ) ;
 
    // compound columns 
    StatusCode column ( const std::string&       name   , 
                        const Hep3Vector&        point  ) ;
    StatusCode column ( const std::string&       name   , 
                        const HepLorentzVector&  vect   ) ;

    // fill few columns at once
    StatusCode fill   ( const char*   format ...        ) ;

    // fill array-like column
    template <class DATA> 
    StatusCode farray ( const std::string& name          , 
                        DATA               first         , 
                        DATA               last          ,
                        const std::string& length        , 
                        const size_t       maxv          ) ;

    // fill array-like column
    template <class DATA> 
    StatusCode farray ( const std::string& name          , 
                        const DATA&        data          , 
                        const std::string& length        , 
                        const size_t       maxv          ) ;
  
    // fill array-like column
    template <class DATA, class FUNCTION> 
    StatusCode farray ( const std::string& name          ,
                        const FUNCTION&    function      ,
                        DATA               first         , 
                        DATA               last          ,
                        const std::string& length        , 
                        const size_t       maxv          ) ; 

    // fill array-like column
    template <class DATA, class FUNC1, class FUNC2> 
    StatusCode farray ( const std::string& name1         ,
                        const FUNC1&       func1         ,
                        const std::string& name2         ,
                        const FUNC2&       func2         ,
                        DATA               first         , 
                        DATA               last          ,
                        const std::string& length        , 
                        const size_t       maxv          ) ; 
  
 \end{lstlisting}
\end{scriptsize}

Few additional methods for filling of several array-like columns 
at once are also exist. They are described in details 
in {\scbf{LoKi}} User Guide and Doxygen documentation for {\scbf{LoKi}}
package.

The full list of algorithm properties, 
introduced for the class {\bftt{GaudiTupleAlgorithm}}
is presented in Table~\ref{Table:AlgTupleProperties}.

\begin{table}[hbt]
  \caption[Standard properties of class {\bftt{GaudiTupleAlg}}]
          {The standard properties of class {\bftt{GaudiTupleAlg}} 
and their default values.}
\bigskip
\label{Table:AlgTupleProperties}
\begin{tabular*}{\linewidth}{@{\hspace{10mm}}l@{\extracolsep{\fill}}l@{\hspace{10mm}}}
\hline 
\hline 
Property Name & Default Value  \\
\hline
{\bftt{"ProduceTuples"}}            &  {\bftt{true}}             \\ 
{\bftt{"TupleOffSet"}}              &  {\bftt{1000}}             \\
{\bftt{"SplitTupleDir"}}            &  {\bftt{true}}             \\
{\bftt{"TupleTopDir"}}              &  {\bftt{""}}               \\
{\bftt{"TupleDir"}}                 &  Algorithm instance name \\
{\bftt{"TupleLUN"}}                 &  {\bftt{"FILE1"}}          \\
\hline 
\end{tabular*} 
\end{table}

The producton of N-tuples can be switched on/off using the  
property {\bftt{"ProduceTuples"}}. 
 For switched off production of tuples 
all methods have no effect. It is especially useful
to use N-tuples for e.g. debugging of 
physics analysis algorithm for preselection   
and to switch off all N-tuples for running of this preselection 
algorithm for large event statistics.

Since in Gaudi there is no principal difference\footnote{There exist only few minor 
specific features, which can be easily taken into account} between 
N-Tuples and Event Tag Collections, the same class 
{\bftt{GaudiTupleAlg}} is used to produce Event Tag Collections, 
which are of especial interest for physics analysis\footnote{The idea of 
having a separate class {\bftt{GaudiTagsAlg}} is not attractive at all, if 
only 1 method need to be modified.}.


\section{Implementation policy for derived classes}
If concrete algorithm requires the implementation of 
the specific {\bftt{initialize}} or 
{\bftt{finalize}} methods they {\bftt{must}} be implemented 
in a following way:

\begin{scriptsize}
 \begin{lstlisting}{}

   class MyAlg: public GaudiTupleAlg { ... } ;

   // standard constructor 
   MyAlg::MyAlg( const std::string& name , 
                 ISvcLocator*       svc  ) 
   : GaudiTupleAlg( name , svc ) 
   , ....

   // standard initialization
   StatusCode MyAlg::initialize() 
   {
     // initialize the base class 
     StatusCode sc = GaudiTupleAlg::initialize() ;
     if( sc.isFailure() ) 
     { return Error("Base class is not initialized",sc);}
     // put specific code here 
     ...
     return StatusCode::SUCCESS ;
   };
   
   // standard finalization
   StatusCode MyAlg::finalization() 
   {
     // put specific code here 
     ...
     return GaudiTupleAlg::finalize();
   };

 \end{lstlisting}
\end{scriptsize}




\end{document}
